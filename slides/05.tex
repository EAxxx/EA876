\documentclass{beamer}
%
% Choose how your presentation looks.
%
% For more themes, color themes and font themes, see:
% http://deic.uab.es/~iblanes/beamer_gallery/index_by_theme.html
%
\mode<presentation>
{
  \usetheme{Madrid}      % or try Darmstadt, Madrid, Warsaw, ...
  \usecolortheme{default} % or try albatross, beaver, crane, ...
  \usefonttheme{default}  % or try serif, structurebold, ...
  \setbeamertemplate{navigation symbols}{}
  \setbeamertemplate{caption}[numbered]
}

\usepackage[english]{babel}
\usepackage[utf8x]{inputenc}
\usepackage{graphicx}
\usepackage{array}

\title[06-Gramática Livre de Contexto]{Gramática Livre de Contexto}
\author{Tiago F. Tavares}
\institute{FEEC -- UNICAMP}
\date{}

\begin{document}

\begin{frame}
  \titlepage
\end{frame}

% Uncomment these lines for an automatically generated outline.
%\begin{frame}{Outline}
%  \tableofcontents
%\end{frame}

\section{Introdução}

\begin{frame}{Revisão...}
  \Large
  \begin{enumerate}
  \item Máquinas de estado podem reconhecer categorias específicas de cadeias de
    caracteres,
  \item Expressões regulares são formas válidas de expressar máquinas de estados
    que reconhecem caracteres.
  \item Posso usar expressões regulares para reconhecer tokens em linguagens de
    programação.
  \end{enumerate}
\end{frame}


\begin{frame}{Objetivos}
  \Large
  \begin{itemize}
    \item Entender quais problemas não podem ser resolvidos por gramáticas
       regulares.
    \item Entender gramáticas livres de contexto como regras de formação.
    \item Entender como resolver expressões matemáticas com GLCs.
  \end{itemize}
\end{frame}


\begin{frame}{Revisão: Lex}
  \Large
  Exercício 1!
\end{frame}


\begin{frame}{Expressões Matemáticas}
\large
  Como é possível saber que $56 + 93$ resulta em um inteiro? Ordene as etapas do
  raciocínio!
  \begin{enumerate}
    \item podemos trocar $56 + 93$ pelo valor $149$
    \item O sinal $+$ denota uma operação matemática.
    \item Na soma, podemos trocar os valores dos operandos e o sinal $+$ pelo
      valor do resultado da adição.
    \item $56$ e $93$ podem ser identificados como números inteiros.
    \item Definimos arbitrariamente a operação $+$ como a adição matemática
      clássica.
    \item $149$ pode ser identificado como inteiro
  \end{enumerate}
\end{frame}

\begin{frame}{Gerando e resolvendo somas}
\large
\centering
  \includegraphics[width=0.8\textwidth]{dot/grafo_soma.pdf}

  Trata-se de uma \textit{regra de formação}.
\end{frame}


\begin{frame}{Gerando e resolvendo somas}
\large
\centering
  \includegraphics[width=0.8\textwidth]{dot/grafo_soma2.pdf}

Para gerar expressões, aplicamos regras no sentido direto. Para resolvê-las,
  aplicamos as regras de formação no sentido reverso.
\end{frame}

\begin{frame}{Gramáticas Livres de Contexto}
\large
\centering
  \begin{itemize}
  \item Regras de formação de strings definem uma gramática!
  \item Não é uma Gramática Regular (não pode ser expressa por expressões
    regulares)
  \item São chamadas ``Livres de Contexto'' porque...
  \item <2-> ... a aplicação das regras em um nível da árvore não depende das
    regras que foram aplicadas nos outros níveis da árvore, apenas dos
      resultados.
  \end{itemize}
\end{frame}

\begin{frame}{Notação}
\Large
  Usar símbolo, seta e então símbolo ``gerado'':
  \begin{enumerate}
  \item $S \rightarrow \phi$
  \item $S \rightarrow S S$
  \item $S \rightarrow A$
  \item $S \rightarrow B$
  \end{enumerate}
  Cada ``regra'' é chamada de ``produção''.
\end{frame}

\begin{frame}{Notação}
\Large
  Símbolos terminais: símbolos do alfabeto que aparecem na definição da GLC.
  \begin{enumerate}
  \item $S \rightarrow \phi$
  \item $S \rightarrow S S$
  \item $S \rightarrow A$
  \item $S \rightarrow B$
  \end{enumerate}
  Neste caso, $\phi$, A e B são símbolos terminais.
\end{frame}

\begin{frame}{Notação}
\Large
  Símbolos não-terminais: símbolos da GLC que aparecem para denotar padrões.
  \begin{enumerate}
  \item $S \rightarrow \phi$
  \item $S \rightarrow S S$
  \item $S \rightarrow A$
  \item $S \rightarrow B$
  \end{enumerate}
  Neste caso, S é um símbolo não-terminal.
\end{frame}

\begin{frame}{Notação}
\Large
  Símbolos inicial: um símbolo não-terminal especial, que denota o símbolo
  ``semente'' (inicial) em uma GLC.
  \begin{enumerate}
  \item $S \rightarrow \phi$
  \item $S \rightarrow S S$
  \item $S \rightarrow A$
  \item $S \rightarrow B$
  \end{enumerate}
  Neste caso, S é um símbolo inicial.
\end{frame}

\begin{frame}{Notação. Exercício!}
\Large
  Identifique os símbolos terminais, não-terminais e o símbolo inicial:
  \begin{enumerate}
  \item $S \rightarrow \phi$
  \item $S \rightarrow S S$
  \item $S \rightarrow A$
  \item $X \rightarrow X S$
  \item $X \rightarrow B$
  \end{enumerate}
\end{frame}

\begin{frame}{Exercício}
\Large
\centering
  Exercício 2
\end{frame}

\begin{frame}{Exercício}
\Large
\centering
  Exercício 3

\end{frame}

\begin{frame}{Autômatos}
\Large
\centering
A expressão regular \textsc{1*} é equivalente à GLC dada por:
  \begin{itemize}
    \item $S \rightarrow \phi$
    \item $S \rightarrow 1 S$
  \end{itemize}

Por que é impossível fazer uma expressão regular equivalente à GLC dada por:
  \begin{itemize}
    \item $S \rightarrow \phi$
    \item $S \rightarrow 1 S 0$
  \end{itemize}

\end{frame}

\begin{frame}{Autômato com Pilha}
\Large
\centering
Exercício 4
\end{frame}




\end{document}
