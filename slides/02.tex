\documentclass{beamer}
%
% Choose how your presentation looks.
%
% For more themes, color themes and font themes, see:
% http://deic.uab.es/~iblanes/beamer_gallery/index_by_theme.html
%
\mode<presentation>
{
  \usetheme{Madrid}      % or try Darmstadt, Madrid, Warsaw, ...
  \usecolortheme{default} % or try albatross, beaver, crane, ...
  \usefonttheme{default}  % or try serif, structurebold, ...
  \setbeamertemplate{navigation symbols}{}
  \setbeamertemplate{caption}[numbered]
}

\usepackage[english]{babel}
\usepackage[utf8x]{inputenc}
\usepackage{graphicx}
\usepackage{array}

\title[03-Arquiteturas]{EA879 -- Introdução ao Software Básico\\Máquinas de
Estado}
\author{Tiago F. Tavares}
\institute{FEEC -- UNICAMP}
\date{Aula 03 -- 15/agosto/2017}

\begin{document}

\begin{frame}
  \titlepage
\end{frame}

% Uncomment these lines for an automatically generated outline.
%\begin{frame}{Outline}
%  \tableofcontents
%\end{frame}

\section{Introdução}

\begin{frame}{Objetivos}
  \Large
  \begin{itemize}
    \item Entender o que são máquinas de estado.
    \item Entender como máquinas de estado podem ser usadas para reconhecer
      cadeias de caracteres.
    \item Aplicar máquinas de estado para reconhecer cadeias de caracteres.
  \end{itemize}
\end{frame}

\begin{frame}{Procedimentos}
  \centering
  \begin{tabular}{l m{6cm}}
    \begin{minipage}{.3\textwidth}
  \includegraphics[scale=0.45]{dot/fluxo1.pdf}
    \end{minipage}
  &
  O que está representado na figura ao lado? Como representamos:
    \begin{itemize}
  \item ações
  \item regras de transição
  \item etapas do processo
    \end{itemize}
  \\
  \end{tabular}
\end{frame}

\begin{frame}{Procedimentos}
  \centering
  \Large
  \begin{tabular}{l m{6cm}}
    \begin{minipage}{.3\textwidth}
  \includegraphics[scale=0.45]{dot/fluxo1.pdf}
    \end{minipage}
  &
    A espera de 5min após adicionar o chá é equivalente à espera de 5min após
    ligar o fogo? Em outras palavras, seria possível que eles fossem
    representados como uma única etapa?
    \\
  \end{tabular}
\end{frame}

\begin{frame}{Máquinas de Estado}
  \centering
  \large
  \begin{tabular}{l m{6cm}}
    \begin{minipage}{.3\textwidth}
  \includegraphics[scale=0.45]{dot/fluxo1.pdf}
    \end{minipage}
  &
    Uma máquina de estados é um conjunto de \textit{estados discretos} que
    descrevem como um sistema deve se comportar ao longo do tempo. No caso ao
    lado:
    \begin{enumerate}
      \item O que cada estado representa?
      \item O que a máquina de estados representa?
      \item De que tipo de entrada as regras de transição dependem?
      \item Que sistema executa a máquina de estados?
    \end{enumerate}
    \\
  \end{tabular}
\end{frame}

\begin{frame}[fragile]{Detectores de sequência}
  \centering
  \large
  \begin{tabular}{l m{8cm}}
    \begin{minipage}{.3\textwidth}
   \includegraphics[scale=0.6]{dot/fluxo2.pdf}
    \end{minipage}
  &
    \begin{verbatim}
    #define TAM_MAX 50
    char entradas[TAM_MAX];
    char c;
    int estado = 0;
    int i = 0;
    for (i=0; i<TAM_ENTRADA; i++) {
      c = entradas[i]
      switch(estado) {
        case 0:
          estado = state0_run(c);
          break;
        ...
      }
    }

    \end{verbatim}
    \\
  \end{tabular}
\end{frame}


\begin{frame}[fragile]{Detectores de sequência}
  \centering
  \large
  \begin{tabular}{l m{8cm}}
    \begin{minipage}{.3\textwidth}
   \includegraphics[scale=0.6]{dot/fluxo2.pdf}
    \end{minipage}
  &
    \begin{verbatim}
    int state0_run(char entrada) {
      switch (entrada) {
        case 'A':
         return 1;
         break;
        case 'B':
         return 0;
         break;
        default:
         return 0;
      }
    }
    \end{verbatim}
    \\
  \end{tabular}
\end{frame}

\begin{frame}[fragile]{Detectores de sequência}
  \centering
  \large
  \begin{tabular}{l m{8cm}}
    \begin{minipage}{.3\textwidth}
   \includegraphics[scale=0.6]{dot/fluxo2.pdf}
    \end{minipage}
  &
    \begin{verbatim}
    int state0_run(char entrada) {
      switch (entrada) {
        case 'A':
         return 1;
         break;
        case 'B':
         return 0;
         break;
        default:
         return 0;
      }
    }
    \end{verbatim}
    \\
  \end{tabular}
\end{frame}

\begin{frame}[fragile]{Detectores de sequência}
  \centering
  \large
  \begin{tabular}{l m{8cm}}
    \begin{minipage}{.3\textwidth}
   \includegraphics[scale=0.6]{dot/fluxo2.pdf}
    \end{minipage}
  &
    O que devem ser as variáveis X, Y e Z no código abaixo:
    \begin{verbatim}
    int state1_run(char entrada) {
      switch (entrada) {
        case 'A':
         return X;
         break;
        case 'B':
         return Y;
         break;
        default:
         return Z;
      }
    }
    \end{verbatim}
    \\
  \end{tabular}
\end{frame}


\begin{frame}[fragile]{Detectores de sequência}
  \centering
  \large
  \begin{tabular}{l m{8cm}}
    \begin{minipage}{.3\textwidth}
   \includegraphics[scale=0.6]{dot/fluxo2.pdf}
    \end{minipage}
  &
    O que devem ser as variáveis X, Y e Z no código abaixo:
    \begin{verbatim}
    int state2_run(char entrada) {
      switch (entrada) {
        case 'A':
         return X;
         break;
        case 'B':
         return Y;
         break;
        default:
         return Z;
      }
    }
    \end{verbatim}
    \\
  \end{tabular}
\end{frame}

\begin{frame}[fragile]{Detectores de sequência}
  \centering
  \large
  \begin{tabular}{l m{8cm}}
    \begin{minipage}{.3\textwidth}
      \includegraphics[scale=0.6]{dot/fluxo2.pdf}
    \end{minipage}
  &
    Como é possível implementar uma função que retorna 1 caso a sequência lida
    seja AB e 0 caso contrário?
    \begin{verbatim}
    int aceitar(int estado) {


    }

    \end{verbatim}
    \\
  \end{tabular}
\end{frame}



\begin{frame}[fragile]{Detectores de sequência}
  \centering
  \large
  \begin{tabular}{l m{8cm}}
    \begin{minipage}{.3\textwidth}
   \includegraphics[scale=0.6]{dot/fluxo2.pdf}
    \end{minipage}
  &
    Nesta máquina de estados:
    \begin{enumerate}
    \item Onde as regras de transição estão mostradas?
    \item Como é possível saber se uma sequência de entradas foi válida?
    \end{enumerate}
    \\
  \end{tabular}
\end{frame}

\begin{frame}[fragile]{Exercício 1}
  \centering
  \large

\end{frame}
\begin{frame}[fragile]{Exercício 2}
  \centering
  \large

\end{frame}

\begin{frame}[fragile]{Exercício 3}
  \centering
  \large

\end{frame}

\begin{frame}[fragile]{Exercício 4}
  \centering
  \large

\end{frame}


\end{document}
