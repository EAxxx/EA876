\documentclass{beamer}
%
% Choose how your presentation looks.
%
% For more themes, color themes and font themes, see:
% http://deic.uab.es/~iblanes/beamer_gallery/index_by_theme.html
%
\mode<presentation>
{
  \usetheme{Madrid}      % or try Darmstadt, Madrid, Warsaw, ...
  \usecolortheme{default} % or try albatross, beaver, crane, ...
  \usefonttheme{default}  % or try serif, structurebold, ...
  \setbeamertemplate{navigation symbols}{}
  \setbeamertemplate{caption}[numbered]
}

\usepackage[english]{babel}
\usepackage[utf8x]{inputenc}
\usepackage{graphicx}
\graphicspath{{./fig/}}

\title[01-Introdução]{EA876 -- Introdução ao Software de Sistema\\Plano de ação}
\author{Tiago F. Tavares}
\institute{FEEC -- UNICAMP}
\date{}

\begin{document}

\begin{frame}
  \titlepage
\end{frame}

% Uncomment these lines for an automatically generated outline.
%\begin{frame}{Outline}
%  \tableofcontents
%\end{frame}

\section{Introdução}

\begin{frame}{Equipe}

\begin{itemize}
  \item Tiago F. Tavares -- sala 311 -- \url{tavares@dca.fee.unicamp.br}
  \item Augusto Fraga -- \url{augustofg96@gmail.com}
\end{itemize}

\vskip 1cm

\begin{block}{Como entrar em contato}
  Mande um e-mail (para o Augusto ou para o Tiago: o Augusto provavelmente vai
  ser mais eficaz em responder dúvidas ``técnicas'') ou bata na porta.
  Os horários de monitoria são quarta e sexta, das 13h00 às 14h00, no bitolódromo.
\end{block}
\end{frame}


\begin{frame}{Site do curso}
\LARGE
Para coordenar nosso curso, utilizaremos o Google Classroom e um repositório no
GitHub.
\end{frame}

\begin{frame}{Horário}
\LARGE
\begin{itemize}
\item Início: 14h00
\item Final: 15h45 ($\pm 15$ minutos)
\item Sala: FE-11
\end{itemize}
\end{frame}

\begin{frame}{Ensino ativo}
\Large
Nesta disciplina, utilizaremos técnicas de ensino ativo inspiradas em
Problem-Based Learning (PBL), adaptadas às condições materiais que temos
atualmente.
\end{frame}

\begin{frame}{}
\includegraphics[width=\textwidth]{deathbypowerpoint.png}
\end{frame}

\begin{frame}{3 verdades, 1 mentira}
\Large
  Fatos sobre o Tiago:
  \begin{enumerate}
    \item Eu tocava baixo numa banda de rock durante a faculdade.
    \item Eu pratico CrossFit.
    \item Certa vez eu pulei de asa delta da Pedra da Gávea no Rio de Janeiro.
    \item Eu tenho muita dificuldade com álgebra.
  \end{enumerate}
\end{frame}

\begin{frame}{Formar grupos}
\Large
  \begin{itemize}
    \item Forme um grupo de 4 pessoas.
    \item Cada membro do grupo deve escrever, num papel, 4 fatos sobre si mesmo,
      sendo 3 deles verdadeiros e 1 falso.
    \item Após, os outros membros do grupo devem tentar descobrir qual é o fato
      falso dentre todos os que foram escritos.
  \end{itemize}
\end{frame}

\begin{frame}{Processos de aprendizado}
\Large
\begin{enumerate}
\item Individualmente, pense em algo que você aprendeu na sua vida e que foi
  muito representativo para você. Escreva isso em um papel.
\item <2-> Compartilhe seu caso com o seu grupo.
\item <3-> Quais elementos do processo de aprendizado o tornaram significativo?
\item <4-> O grupo consegue identificar elementos que são comuns para todos?
\end{enumerate}
\end{frame}

  \begin{frame}{Ementa}
    \Large
    Atividade: leitura da ementa e discussão do calendário.

  \end{frame}


\end{document}
