\documentclass{beamer}
%
% Choose how your presentation looks.
%
% For more themes, color themes and font themes, see:
% http://deic.uab.es/~iblanes/beamer_gallery/index_by_theme.html
%
\mode<presentation>
{
  \usetheme{Madrid}      % or try Darmstadt, Madrid, Warsaw, ...
  \usecolortheme{default} % or try albatross, beaver, crane, ...
  \usefonttheme{default}  % or try serif, structurebold, ...
  \setbeamertemplate{navigation symbols}{}
  \setbeamertemplate{caption}[numbered]
}

\usepackage[english]{babel}
\usepackage[utf8x]{inputenc}
\usepackage{graphicx}
\usepackage{array}

\title[04-Expressões Regulares]{Expressões Regulares}
\author{Tiago F. Tavares}
\institute{FEEC -- UNICAMP}
\date{}

\begin{document}

\begin{frame}
  \titlepage
\end{frame}

% Uncomment these lines for an automatically generated outline.
%\begin{frame}{Outline}
%  \tableofcontents
%\end{frame}

\section{Introdução}

\begin{frame}{Revisão...}
  \Large
  \begin{enumerate}
  \item Máquinas de estado podem reconhecer categorias específicas de cadeias de
    caracteres
  \end{enumerate}
\end{frame}

\begin{frame}{Objetivos}
  \Large
  \begin{itemize}
  \item Entender o que são gramáticas regulares.
  \item Entender como gramáticas regulares podem ser usadas para reconhecer
    cadeias de caracteres.
  \item Usar sintaxe POSIX de gramáticas regulares
  \item Entender a relação um-para-um entre gramáticas regulares e máquinas de
  estado
  \item Aplicar gramáticas regulares para reconhecer cadeias de caracteres.
  \end{itemize}
\end{frame}

\begin{frame}{Exercício}
  \centering
  No texto abaixo, identifique os verbos no gerúndio, os números inteiros e os
  usernames do twitter:

  \vspace{0.5in}

  \textit{Passei 2 horas upando o código do @maddoghall}.
\end{frame}

\begin{frame}{Reflexão (I)}
  \centering
  \Large
  Pense no verbo no gerúndio: \textit{upando}. Ele não está no dicionário (ou ao
  menos não estava até bem pouco tempo atrás). Por que, mesmo sem estar no
  dicionário, é possível saber que é um verbo no gerúndio?
\end{frame}

\begin{frame}{Máquinas de Estado}
  \centering
  \large
  Proponha uma máquina de estados capaz de reconhecer palavras escritas no
  gerúndio.
\end{frame}


\begin{frame}{Expressões Regulares}
  \centering
  \large
  Uma expressão regular é uma ferramenta para especificar padrões em textos
  usando uma sequência de símbolos. A sintaxe desses símbolos é, a priori,
  livre, mas é muito comum usar a especificação POSIX.
\end{frame}

\begin{frame}{Live coding}
  \centering
  \large
  Vamos usar ls e grep para encontrar nomes de arquivos no meu computador. Vamos
  encontrar, inicialmente, nomes de arquivos que contém a expressão ``tex''.
\end{frame}


\begin{frame}[fragile]{Expressões Regulares}
\centering
\large
  \begin{tabular}{p{5cm} | p{5cm} }
    Significado & Sintaxe POSIX \\ \hline
    Caractere específico & O próprio caractere -- A, a, C, etc. \\ \hline
    Qualquer caractere do grupo & Colchetes -- [ABCDE] \\ \hline
    Qualquer letra minúscula & A-Z (use [A-Za-z] para caixa alta e baixa) \\ \hline
    Uma sequência & Parênteses -- (LAIA) \\ \hline
    Qualquer caractere & . (ponto final) \\ \hline
  \end{tabular}

  Exercícios 1 e 2!
\end{frame}

\begin{frame}[fragile]{Expressões Regulares}
  \centering
  \large
  \begin{tabular}{p{5cm} | p{5cm} }
    Significado & Sintaxe POSIX \\ \hline
    0 ou mais repetições do símbolo anterior & Use * depois do símbolo
    \\ \hline
    1 ou mais repetições do símbolo anterior & Use + depois do símbolo \\ \hline
    \\
    0 ou 1 repetições do símbolo anterio & Use ? depois do símbolo \\ \hline
  \end{tabular}

  Exercícios 3, 4 e 5!

\end{frame}

\begin{frame}{Live coding}
  \centering
  \large
  Vamos usar ls e grep para encontrar nomes de arquivos no meu computador. Junto
  ao seu grupo, proponha expressões regulares para todos esses casos:
  \begin{enumerate}
    \item Arquivos com a extensão ``.tex''
    \item Arquivos começando em ``s'' e com a extensão ``.tex''
    \item Arquivos que tenham algum número no nome
    \item Arquivos que tenham a expressão ``slides'' contida no nome
    \item Arquivos chamados estritamente ``[inteiro].tex''
  \end{enumerate}
\end{frame}

\begin{frame}
  \large
  \begin{tabular}{p{5cm} | p{5cm} }
    Significado & Sintaxe POSIX \\ \hline
    Âncora no começo da linha & $\hat{ }${regex} \\ \hline
    Âncora no final da linha & {regex}\$ \\ \hline
    \\
    Negação & [$\hat{ }$A-Z] (rejeita [A-Z]) \\ \hline
  \end{tabular}
  Estudo de caso: como encontrar nomes dos personagens no texto de Romeu e
  Julieta?
\end{frame}

\begin{frame}{Exercício 6}
  \centering
  \Large
\end{frame}


\end{document}
